\documentclass[onecolumn,10pt]{jhwhw}

\usepackage{epsfig} %% for loading postscript figures
\usepackage{amsmath}
\usepackage{graphicx}
\usepackage{grffile}
\usepackage{pdfpages}
\usepackage{algpseudocode}
\usepackage{wrapfig}
\usepackage{pgfplots}
\usepackage{amsfonts}
\usepackage{booktabs}
\usepackage{siunitx}
\usepackage{commath}
\usepackage{rotating}

% Default fixed font does not support bold face
\DeclareFixedFont{\ttb}{T1}{txtt}{bx}{n}{12} % for bold
\DeclareFixedFont{\ttm}{T1}{txtt}{m}{n}{12}  % for normal

% Custom colors
\usepackage{color}
\usepackage{listings}
\usepackage{framed}
\usepackage{caption}
\usepackage{bm}
\captionsetup[lstlisting]{font={small,tt}}

\definecolor{mygreen}{rgb}{0,0.6,0}
\definecolor{mygray}{rgb}{0.5,0.5,0.5}
\definecolor{mymauve}{rgb}{0.58,0,0.82}

\lstset{ %
  backgroundcolor=\color{white},   % choose the background color; you must add \usepackage{color} or \usepackage{xcolor}
  basicstyle=\ttfamily\footnotesize, % the size of the fonts that are used for the code
  breakatwhitespace=false,         % sets if automatic breaks should only happen at whitespace
  breaklines=true,                 % sets automatic line breaking
  captionpos=b,                    % sets the caption-position to bottom
  commentstyle=\color{mygreen},    % comment style
  deletekeywords={...},            % if you want to delete keywords from the given language
  escapeinside={\%*}{*)},          % if you want to add LaTeX within your code
  extendedchars=true,              % lets you use non-ASCII characters; for 8-bits encodings only, does not work with UTF-8
  frame=single,                    % adds a frame around the code
  keepspaces=true,                 % keeps spaces in text, useful for keeping indentation of code (possibly needs columns=flexible)
  columns=flexible,
  keywordstyle=\color{blue},       % keyword style
  language=Python,                 % the language of the code
  morekeywords={*,...},            % if you want to add more keywords to the set
  numbers=left,                    % where to put the line-numbers; possible values are (none, left, right)
  numbersep=5pt,                   % how far the line-numbers are from the code
  numberstyle=\tiny\color{mygray}, % the style that is used for the line-numbers
  rulecolor=\color{black},         % if not set, the frame-color may be changed on line-breaks within not-black text (e.g. comments (green here))
  showspaces=false,                % show spaces everywhere adding particular underscores; it overrides 'showstringspaces'
  showstringspaces=false,          % underline spaces within strings only
  showtabs=false,                  % show tabs within strings adding particular underscores
  stepnumber=1,                    % the step between two line-numbers. If it's 1, each line will be numbered
  stringstyle=\color{mymauve},     % string literal style
  tabsize=4,                       % sets default tabsize to 2 spaces
}

\author{John Karasinski}
\title{Homework 2}

\begin{document}
%\maketitle

\problem{Trapped Particle Radiation}
Assume two spacecraft orbit Earth, one at an altitude of 3000km, the other at 20,000km.
\part{Remind yourself --- are these orbits above or below geostationary?}
\part{Assuming they are both shielded with 4mm aluminum and have the same electronics, which spacecraft is more likely to suffer Single-Event Upsets? Why?}

\problem{Upper Atmosphere Physics}
\part{What is the change in density of the exosphere at Hubble Space Telescope (HST) altitude between the date of its fourth Shuttle re-boost (Servicing Mission 3B) and the date of your new spacecraft’s re-boost?}
\part{Using the NASA NRLMSISE-100 atmosphere model, list the following parameters for today’s date and the HST altitude:}
\begin{enumerate}
\item Atomic oxygen species density
\item Atmosphere mass density
\item Exosphere temperature
\end{enumerate}
\part{What role do each of these environmental parameters play in the mission design of your spacecraft?
}

\problem{Orbital Debris and Ballistic Limit Equations (BLE)}
\part{Using eqn 2-3 of the NASA MMOD handbook and Fig 2-4 of the NASA Orbital Debris Engineering Model (both in MMOD folder on SmartSite), determine the critical particle diameter for a Probability of No Penetration (PNP) of 0.877. Assume 1m$^2$ exposed area and
a two-week mission.}
\part{What would the critical particle diameter be for the same PNP if your spacecraft stayed attached to HST for a second re-boost in 2 years?}
\part{Consider a perpendicular impact of a particle of the size found in part a and another particle with 1mm diameter. Use the Design Equations 4-1 and 4-2 to compute the penetration depth for:}
\begin{enumerate}
\item Silicate particle at 0.5gm/cm$^3$ and 23 km/sec (micro-meteroid)
\item Steel particle at 7 km/sec (orbital debris)
\end{enumerate}
With target materials of:
\begin{enumerate}
\item 6061-T6 Alum
\item 7075-T6 Alum
\end{enumerate}
\part{No protection case: Use Performance Equation 4-6 to estimate the required spacecraft wall thickness to avoid detached spall due to impact of the 1mm particle for the four material cases in part c.}
\part{Compare your answers in part d to equation 4-4.}
\part{Whipple Shield design: For the two 1mm particle compositions/velocities in part c, use equations 4-21 and 4-22 to estimate the bumper and rear-wall thickness required to defeat the threat particles. Assume}
\begin{enumerate}
\item Particles are spherical
\item Bumper standoff = 10.2cm (Fig 4-1)
\item Rear wall: 2219-T87 Alum, 0.5cm thickness
\item Perpendicular velocity impact ($\theta=0$)
\end{enumerate}
\part{Estimate the protection capability limits for your Whipple Shields: Compute the critical projectile performance diameter using equations 4-23, 4-24, and 4-25 for various relative velocities, for two types of spherical particle, one silicate, the other steel. Assume the same parameters as for part f.}

\problem{Acoustic Shielding}
\part{Download and install a sound-level app on your smartphone.}
\part{Measure the sound pressure level (dB) in some noisy environment (home stereo? EFL?).}
\part{Remind yourself: what is the definition of the Decibel, and how is the reference soundpressure level defined?}
\part{Use a Styrofoam cup (or similar) as a payload shroud for your phone. Plug the open end of the shroud with isolation (tissues?). Measure the change in dB sensed by your phone.}
\part{Add some kind of insulation to the inside walls of your payload shroud and repeat part d.}
\part{Discuss results.}

\problem{Numerical Integration Review}
Consider the first-order initial-value problem:
$\dfrac{dy}{dt} = t + y$, $y(0)=0$ with exact solution $y(t) = e^t-t-1$
\part{Program Euler and Runge-Kutta solvers (write your own) and plot the results over a range t=0 to 1.0, with step sizes h=0.01, 0.1, and 0.5. Plot results (y vs t) and compare to the exact solution.}
\part{Repeat step a with library functions for Euler and RK4}

\end{document}
