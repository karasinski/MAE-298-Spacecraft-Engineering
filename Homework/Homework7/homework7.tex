\documentclass[onecolumn,10pt]{jhwhw}

\usepackage{epsfig} %% for loading postscript figures
\usepackage{amsmath}
\usepackage{graphicx}
\usepackage{grffile}
\usepackage{pdfpages}
\usepackage{algpseudocode}
\usepackage{wrapfig}
\usepackage{pgfplots}
\usepackage{amsfonts}
\usepackage{booktabs}
\usepackage{siunitx}
\usepackage{commath}
\usepackage{rotating}
\usepackage{url}
\usepackage{multimedia}
\usepackage{hyperref}
\usepackage{mathtools}

% Default fixed font does not support bold face
\DeclareFixedFont{\ttb}{T1}{txtt}{bx}{n}{12} % for bold
\DeclareFixedFont{\ttm}{T1}{txtt}{m}{n}{12}  % for normal

% Custom colors
\usepackage{color}
\usepackage{listings}
\usepackage{framed}
\usepackage{caption}
\usepackage{bm}
\captionsetup[lstlisting]{font={small,tt}}

\definecolor{mygreen}{rgb}{0,0.6,0}
\definecolor{mygray}{rgb}{0.5,0.5,0.5}
\definecolor{mymauve}{rgb}{0.58,0,0.82}

\lstset{ %
  backgroundcolor=\color{white},   % choose the background color; you must add \usepackage{color} or \usepackage{xcolor}
  basicstyle=\ttfamily\footnotesize, % the size of the fonts that are used for the code
  breakatwhitespace=false,         % sets if automatic breaks should only happen at whitespace
  breaklines=true,                 % sets automatic line breaking
  captionpos=b,                    % sets the caption-position to bottom
  commentstyle=\color{mygreen},    % comment style
  deletekeywords={...},            % if you want to delete keywords from the given language
  escapeinside={\%*}{*)},          % if you want to add LaTeX within your code
  extendedchars=true,              % lets you use non-ASCII characters; for 8-bits encodings only, does not work with UTF-8
  frame=single,                    % adds a frame around the code
  keepspaces=true,                 % keeps spaces in text, useful for keeping indentation of code (possibly needs columns=flexible)
  columns=flexible,
  keywordstyle=\color{blue},       % keyword style
  language=Python,                 % the language of the code
  morekeywords={*,...},            % if you want to add more keywords to the set
  numbers=left,                    % where to put the line-numbers; possible values are (none, left, right)
  numbersep=5pt,                   % how far the line-numbers are from the code
  numberstyle=\tiny\color{mygray}, % the style that is used for the line-numbers
  rulecolor=\color{black},         % if not set, the frame-color may be changed on line-breaks within not-black text (e.g. comments (green here))
  showspaces=false,                % show spaces everywhere adding particular underscores; it overrides 'showstringspaces'
  showstringspaces=false,          % underline spaces within strings only
  showtabs=false,                  % show tabs within strings adding particular underscores
  stepnumber=1,                    % the step between two line-numbers. If it's 1, each line will be numbered
  stringstyle=\color{mymauve},     % string literal style
  tabsize=4,                       % sets default tabsize to 2 spaces
}

\usepackage{etoolbox}
\renewcommand{\lstlistingname}{Diagram}% Listing -> Algorithm
\patchcmd{\thebibliography}{\chapter*}{\section*}{}{}

\author{John Karasinski}
\title{Homework 7}

\begin{document}
%\maketitle

\problem{}
\textit{Consider a spacecraft in a circular LEO at altitude of 200km. Design a system diagram and estimate the mass and volume of a propulsion system to transfer to a circular orbit of altitude 350km. Consider your valving to protect for jet fail on/jet fail cases, as appropriate. Include propellant and N2 pressurant tanks, plumbing, valves, and a single engine for:}
\begin{enumerate}
\item Monopropellant (hydrazine)
\item Bi-propellant (your choice)
\end{enumerate}

The amount of $\Delta V$ required for a Hohmann transfer from a 200km to a 350km orbit, under the assumption of instantaneous impulses is
\begin{align*}
\Delta v_1 = \sqrt{\frac{\mu}{r_1}} \left( \sqrt{\frac{2 r_2}{r_1+r_2}} - 1 \right),
\end{align*}
to enter the elliptical orbit at $r=r_1$ from the $r_1$ circular orbit, and
\begin{align*}
\Delta v_2  = \sqrt{\frac{\mu}{r_2}} \left( 1 - \sqrt{\frac{2 r_1}{r_1+r_2}}\,\,\right),
\end{align*}
to leave the elliptical orbit at $r=r_2$ to the $r_2$ circular orbit. For $r_1 = 6571$km and $r_2 = 6721$km, with $\mu=398600.4$km$^3$s$^{-2}$, the total $\Delta V$ required is 87.3 m s$^{-1}$.

\begin{table}[h]
\begin{center}
\begin{tabular}{*{8}{r}}
\toprule
Engine & Type     & Mass (kg) & Propellant & Thrust (N) & SI (s) \\
\midrule
MR-80B & Monoprop & 8.5 &       Hydrazine & 3,780 & 225 \\
R-40B  & Biprop   & 6.8 & NTO (MON-3)/MMH & 4,000 & 293 \\
\bottomrule
\end{tabular}
\end{center}
\caption{Information on the two engines used for this analysis.}
\end{table}

To calculate our change in mass, we can use the rocket equation,
\begin{align*}
\dfrac{M_i}{M_f} &= exp({\dfrac{\Delta V}{g_0 \cdot \mbox{ISP}}}) \\
\dfrac{M_{fuel} + M_f}{M_f} &= exp({\dfrac{\Delta V}{g_0 \cdot \mbox{ISP}}}) \\
M_{fuel} &= M_f \left( exp({\dfrac{\Delta V}{g_0 \cdot \mbox{ISP}}}) -1 \right),
\end{align*}
which shows that the required $\Delta V$ is a function of the initial mass of the craft. For a small satellite with a mass of 500kg, the required mass for fuel is listed in Table~\ref{fuel_table}.

\begin{table}[h]
\begin{center}
\begin{tabular}{*{8}{r}}
\toprule
Engine & SI (s) & $\Delta V$ (kg) & Volume (m$^3$)\\
\midrule
MR-80B & 225 & 20.19 & 0.0200 \\
R-40B  & 293 & 15.43 & 0.0128 \\
\bottomrule
\end{tabular}
\end{center}
\caption{Information on the two engines used for this analysis.}
\label{fuel_table}
\end{table}

To calculate the volume required for the burn, we can simply calculate $V = m/\rho$. Hydrazine has a density of 1011 kg m$^{-3}$, and for 20.19 kg of fuel this results in a required volume of 0.0200 $m^3$. For the bipropellant, we need to mix fuel and an oxidizer at a mass ratio of MON/MMH$ = 2.16$. This results in 10.54 kg of fuel, and 4.88 kg of oxidizer. MON has a density of 1442 kg m$^{-3}$, and MMH has a density of 880 kg m$^{-3}$. This results in tank volumes of .0073 m$^3$ and .0055 m$^3$, respectively.

\begin{lstlisting}[caption={Serial and parallel valve designs for monoprop and bipropellant systems. Each x represents a valve.}]
########################################
#              +-----+                 #
#              | Fuel|                 #
#              +--+--+                 #
#                 |                    #
#              x--+--x                 #
#              |     |                 #
#              |     |                 #
#              |     |                 #
#              x--+--x                 #
#                 |                    #
#                 |                    #
#                 |                    #
#                 |                    #
#                 +                    #
#                / \                   #
#               /   \                  #
#              /     \                 #
#              -------                 #
########################################
#      +-----+         +-----+         #
#      | Fuel|         | Oxid|         #
#      +--+--+         +--+--+         #
#         |               |            #
#      x--+--x         x--+--x         #
#      |     |         |     |         #
#      |     |         |     |         #
#      |     |         |     |         #
#      x--+--x         x--+--x         #
#         |               |            #
#         +-------+-------+            #
#                 |                    #
#                 |                    #
#                 +                    #
#                / \                   #
#               /   \                  #
#              /     \                 #
#              -------                 #
########################################
\end{lstlisting}


\problem{}
\textit{Read ``Mission Analysis for a Micro RF Ion Thruster for CubeSat Orbital Maneuvers''}
\begin{enumerate}
\item Which propulsion system included in the paper would you choose for a maximum orbit change given mass and volume constraints?

Busek Green Monopropellant 3U

\item What is the chemical name and estimated ISP for the propellant of the system chosen in a)?

Chemical name: AF-M315E, ISP: 230 s

\item If you wanted to minimize propellant usage for the circularization burn of a Hohmann transfer, where on the orbit would you burn?

Perigee

\item For use on a 3U CubeSat, what is the total mass and volume of the thruster, propellant, and tank chosen in a)?

Mass of thruster: $\textless$ 0.7 kg, Volume of thruster: .0514 U \\
Mass of propellant: 4.9959 kg, Volume of propellant: 3.3986 U \\
Mass of tank: 2 kg, Volume of tank: 3.3986 U

\item What is the nominal thrust level and input power required? Why is input power required at all?

Thrust: 500 mN, Inputer Power: 20 W \\
Power is required for both the valve, and for preheating. The thrusters have a minimum start temperature.

``Due to the advanced monopropellant thrusters’ elevated minimum start temperature, catalyst bed preheat power requirements are higher compared to a conventional hydrazine system. This increase is partially offset, however, by the reduced power needs of the thrusters’ single seat valves, as well as much lower power required for system thermal management during non-operating periods enabled by the propellant’s demonstrated storage stability very low temperatures (although current CONOPS for the GPIM mission call for the propellant to be maintained within nominal system operating range).'' Spores, Ronald A., et al. ``GPIM AF-M315E propulsion system.'' 49th AIAA Joint PropulsionConference (2013).

\item Why is the propellant you have chosen any better than hydrazine?

The green propellant can be handled without a hazmat suit, and won't kill anyone after accidental exposure. Additionally, AF-M315E offers a 12\% higher ISP, and is 45\% more dense. Additionally, this fuel cannot freeze, and does not require constant heating.

\item What is the main reason that you cannot instead use the Aerojet/Rocketdyne MPS-110 Cold-Gas Thruster system for CubeSats?

The MPS-110 is still in development, and is not yet available for purchase.

\end{enumerate}

\problem{}
\textit{Read ``Using Additive Manufacturing to Print a CubeSat Propulsion System''}
\begin{enumerate}
\item One problem encountered was arcing between a ground wire and the thruster sheath, and of course you worry about thermal containment with a spark-powered thruster in a plastic spacecraft. What would be your recommendation of a less challenging thruster system to study for incorporation into a 3-D printed CubeSat bus? Pros and cons?
\end{enumerate}
\end{document}
