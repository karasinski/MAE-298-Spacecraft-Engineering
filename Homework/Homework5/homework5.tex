\documentclass[onecolumn,10pt]{jhwhw}

\usepackage{epsfig} %% for loading postscript figures
\usepackage{amsmath}
\usepackage{graphicx}
\usepackage{grffile}
\usepackage{pdfpages}
\usepackage{algpseudocode}
\usepackage{wrapfig}
\usepackage{pgfplots}
\usepackage{amsfonts}
\usepackage{booktabs}
\usepackage{siunitx}
\usepackage{commath}
\usepackage{rotating}
\usepackage{url}
\usepackage{multimedia}
\usepackage{hyperref}
\usepackage{mathtools}

% Default fixed font does not support bold face
\DeclareFixedFont{\ttb}{T1}{txtt}{bx}{n}{12} % for bold
\DeclareFixedFont{\ttm}{T1}{txtt}{m}{n}{12}  % for normal

% Custom colors
\usepackage{color}
\usepackage{listings}
\usepackage{framed}
\usepackage{caption}
\usepackage{bm}
\captionsetup[lstlisting]{font={small,tt}}

\definecolor{mygreen}{rgb}{0,0.6,0}
\definecolor{mygray}{rgb}{0.5,0.5,0.5}
\definecolor{mymauve}{rgb}{0.58,0,0.82}

\lstset{ %
  backgroundcolor=\color{white},   % choose the background color; you must add \usepackage{color} or \usepackage{xcolor}
  basicstyle=\ttfamily\footnotesize, % the size of the fonts that are used for the code
  breakatwhitespace=false,         % sets if automatic breaks should only happen at whitespace
  breaklines=true,                 % sets automatic line breaking
  captionpos=b,                    % sets the caption-position to bottom
  commentstyle=\color{mygreen},    % comment style
  deletekeywords={...},            % if you want to delete keywords from the given language
  escapeinside={\%*}{*)},          % if you want to add LaTeX within your code
  extendedchars=true,              % lets you use non-ASCII characters; for 8-bits encodings only, does not work with UTF-8
  frame=single,                    % adds a frame around the code
  keepspaces=true,                 % keeps spaces in text, useful for keeping indentation of code (possibly needs columns=flexible)
  columns=flexible,
  keywordstyle=\color{blue},       % keyword style
  language=Python,                 % the language of the code
  morekeywords={*,...},            % if you want to add more keywords to the set
  numbers=left,                    % where to put the line-numbers; possible values are (none, left, right)
  numbersep=5pt,                   % how far the line-numbers are from the code
  numberstyle=\tiny\color{mygray}, % the style that is used for the line-numbers
  rulecolor=\color{black},         % if not set, the frame-color may be changed on line-breaks within not-black text (e.g. comments (green here))
  showspaces=false,                % show spaces everywhere adding particular underscores; it overrides 'showstringspaces'
  showstringspaces=false,          % underline spaces within strings only
  showtabs=false,                  % show tabs within strings adding particular underscores
  stepnumber=1,                    % the step between two line-numbers. If it's 1, each line will be numbered
  stringstyle=\color{mymauve},     % string literal style
  tabsize=4,                       % sets default tabsize to 2 spaces
}

\usepackage{etoolbox}
\renewcommand{\lstlistingname}{Diagram}% Listing -> Algorithm
\patchcmd{\thebibliography}{\chapter*}{\section*}{}{}

\author{John Karasinski}
\title{Homework 5}

\begin{document}
%\maketitle

\problem{}
\textit{Orbital Transfer Review: remind yourself: (discuss/justify decisions)} \\
\\
All three of the transfers below can be accomplished with a Hohmann transfer. The amount of $\Delta V$ required for a Hohmann transfer is
\begin{align*}
\Delta v_1 &= \sqrt{\dfrac{\mu}{r_1}} \left(\sqrt{\dfrac{2r_2}{r_1 + r_2}} -1 \right), \\
\Delta v_2 &= \sqrt{\dfrac{\mu}{r_2}} \left(1 - \sqrt{\dfrac{2r_1}{r_1 + r_2}} \right), \\
\Delta v_{total} &= \Delta v_1 + \Delta v_2.
\end{align*}
Additionally, part (b) requires an inclination change. The $\Delta V$ budget for a inclination change for a circular orbit is
\begin{align*}
\Delta{v_i}= 2 v \sin \left( \dfrac{\Delta i}{2} \right).
\end{align*}
The assumption of a circular orbit should be fine here, as both HST and ISS orbits have very low eccentricity.

\begin{enumerate}
\itemsep0em
\item Determine the $\Delta V$ required to move from a 200km coplanar parking orbit to the HST orbit
\begin{align*}
r_1 &= 200 km + 6,371 km \\
r_2 &= 569 km + 6,371 km \\
\Delta v_{total} &= \Delta v_{Hohmann} \\
                 &= 0.210 km/s.
\end{align*}
\item Determine the $\Delta V$ required to move from the ISS orbit to the HST orbit
\begin{align*}
r_{ISS} &= 414.1 km + 6,371 km \\
r_{HST} &= 569 km + 6,371 km \\
i_{ISS} &= 0.9014 rad \\
i_{HST} &= 0.4969 rad \\
v_{HST} &= 7.59 km/s \\
\\
\Delta v_{total} &= \Delta v_{Hohmann} + \Delta v_{Plane Change} \\
                 &= 0.086 + 3.049 \\
                 &= 3.135 km/s,
\end{align*}
where we've again assumed a circular orbit. The plane change should take place after the Hohmann transfer, as the orbital velocity is lower at HST orbit, leading to a smaller required $\Delta V$.

\item Determine the $\Delta V$ required to deorbit from the HST orbit (must choose your de-orbit orbital params)
\begin{align*}
r_{deorbit} &= 100 km + 6,371 km \\
r_{HST} &= 569 km + 6,371 km \\
\Delta v &= \sqrt{\dfrac{\mu}{r_1}} \left(\sqrt{\dfrac{2r_2}{r_1 + r_2}} -1 \right), \\
         &= 0.136 km/s,
\end{align*}
where a height of $100km$ should be sufficient to cause the vehicle to deorbit rapidly.
\end{enumerate}

\problem{}
\textit{Eclipse durations (text 5.3.2): an important design aspect of your solar array system is the relative durations of eclipse and insolation. Using the algorithm given in section 5.3.2,}
\begin{enumerate}
\itemsep0em
\item Compute the eclipse period for ISS, for HST, and for a typical GPS satellite (choose one)
\end{enumerate}

I ran the algorithm in section 5.3.2 over one complete orbit at 0.1 degree increments, and for sun positions of one complete year at day increments. The mean and standard error of the mean was calculated for eclipse period, as was the total time in eclipse and the total number of eclipses over the one year period. These results are presented in Table~\ref{p2table}.

\begin{table}[h]
\begin{center}
\begin{tabular}{rrrrr}
\toprule
Satellite & Mean (min) & SEM (min) & Total (hrs) & Count \\
\midrule
HST & 33.97 & 0.10 & 414.45 & 732 \\
ISS & 34.99 & 0.09 & 426.98 & 732 \\
GPS & 46.09 & 2.55 &  90.64 & 118 \\
\bottomrule
\end{tabular}
\end{center}
\caption{Eclipse statistics for the HST, ISS, and a GPS satellite (Navstar 43)}
\label{p2table}
\end{table}



\clearpage

\problem{}
\textit{Let’s say we lose control of your spacecraft after it has undocked from HST, but before it has de-orbited.}
\begin{enumerate}
\itemsep0em
\item Estimate the orbital lifetime of your spacecraft (text 5.3.4) following loss of communications: assume HST
circular orbit, average solar activity.
\item Would it make any difference to the decay timescale whether your spacecraft was tumbling or not?
\end{enumerate}
\clearpage

\problem{}
\textit{Geostationary orbits (text 5.6):}
\begin{enumerate}
\itemsep0em
\item Using the linearized solution to Kepler’s equation given in eqns 5.27-5.29, plot ground-track fluctuations as longitude vs latitude. Describe the results.
\item Define deadband and control limit-cycle in the context of GEO station keeping.
\item Consider a GEO satellite with nominal longitude of -100deg, and an onboard propellant system capable of
providing a total $\Delta V$ of 200m/s. For a maximum longitudinal error magnitude of 0.22deg, for how long can the satellite station-keep?
\end{enumerate}
\clearpage

\problem{}
\textit{Two spacecraft in elliptical Earth orbit with the orbital parameters as follows. Compute the relative position and velocity vectors.}
\begin{enumerate}
\itemsep0em
\item h= 52,059 km$^2$/s, e=0.0257240, i=60deg, $\Omega$=40deg, $\omega$=30deg, $\theta$=40deg
\item h= 52,362 km$^2$/s, e=0.0072696, i=50deg, $\Omega$=40deg, $\omega$=120deg, $\theta$=40deg
\end{enumerate}
\clearpage

\problem{}
\textit{Fly-around relative trajectories: for the lost EVA toolbox example considered in lecture, generate the relative motion plot for 1 orbital period, given initial conditions of:}
\begin{enumerate}
\itemsep0em
\item Release relative velocity = (-0.1, 0, 0) m/s (prolate cycloid)
\item Release relative velocity = (0, 0, 0.1) m/s (ellipse)
\item Release relative velocity = (-0.1, 0, 0.1) m/s (initially 45deg backwards and up; describe subsequent motion)
\item For a and b, plot the trajectory with and without the nt$\ll$1 assumption. Discuss.
\item How about a release relative velocity = (0, 0.1, 0) m/s? Would you see the toolbox again or not?
\end{enumerate}
\clearpage

\problem{}
\textit{For your HST re-boost spacecraft, assume:}
\begin{itemize}
\itemsep0em
\item[--] Launch: drop-off circular orbit at 200km, in-plane with HST, 65deg phase angle behind HST - Phasing: 4-orbit phasing to point S1, 30km behind and 10km below HST
\item[--] Homing: Hohmann S1 to co-orbit waiting point S2, 1km behind HST
\item[--] Closing: Cycloid close waiting point S3, 200m behind HST
\end{itemize}
\begin{enumerate}
\itemsep0em
\item compute the required $\Delta V$ and elapsed time for each phase, and for the total rendezvous to S3
\item compute the view-angle to HST, measured from the orbit-tangent (for sensor acquisition)
\item plot the total quantitative relative motion (like Walter Fig. 8.26)
\end{enumerate}

\end{document}
