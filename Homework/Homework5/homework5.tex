\documentclass[onecolumn,10pt]{jhwhw}

\usepackage{epsfig} %% for loading postscript figures
\usepackage{amsmath}
\usepackage{graphicx}
\usepackage{grffile}
\usepackage{pdfpages}
\usepackage{algpseudocode}
\usepackage{wrapfig}
\usepackage{pgfplots}
\usepackage{amsfonts}
\usepackage{booktabs}
\usepackage{siunitx}
\usepackage{commath}
\usepackage{rotating}
\usepackage{url}
\usepackage{multimedia}
\usepackage{hyperref}
\usepackage{mathtools}

% Default fixed font does not support bold face
\DeclareFixedFont{\ttb}{T1}{txtt}{bx}{n}{12} % for bold
\DeclareFixedFont{\ttm}{T1}{txtt}{m}{n}{12}  % for normal

% Custom colors
\usepackage{color}
\usepackage{listings}
\usepackage{framed}
\usepackage{caption}
\usepackage{bm}
\captionsetup[lstlisting]{font={small,tt}}

\definecolor{mygreen}{rgb}{0,0.6,0}
\definecolor{mygray}{rgb}{0.5,0.5,0.5}
\definecolor{mymauve}{rgb}{0.58,0,0.82}

\lstset{ %
  backgroundcolor=\color{white},   % choose the background color; you must add \usepackage{color} or \usepackage{xcolor}
  basicstyle=\ttfamily\footnotesize, % the size of the fonts that are used for the code
  breakatwhitespace=false,         % sets if automatic breaks should only happen at whitespace
  breaklines=true,                 % sets automatic line breaking
  captionpos=b,                    % sets the caption-position to bottom
  commentstyle=\color{mygreen},    % comment style
  deletekeywords={...},            % if you want to delete keywords from the given language
  escapeinside={\%*}{*)},          % if you want to add LaTeX within your code
  extendedchars=true,              % lets you use non-ASCII characters; for 8-bits encodings only, does not work with UTF-8
  frame=single,                    % adds a frame around the code
  keepspaces=true,                 % keeps spaces in text, useful for keeping indentation of code (possibly needs columns=flexible)
  columns=flexible,
  keywordstyle=\color{blue},       % keyword style
  language=Python,                 % the language of the code
  morekeywords={*,...},            % if you want to add more keywords to the set
  numbers=left,                    % where to put the line-numbers; possible values are (none, left, right)
  numbersep=5pt,                   % how far the line-numbers are from the code
  numberstyle=\tiny\color{mygray}, % the style that is used for the line-numbers
  rulecolor=\color{black},         % if not set, the frame-color may be changed on line-breaks within not-black text (e.g. comments (green here))
  showspaces=false,                % show spaces everywhere adding particular underscores; it overrides 'showstringspaces'
  showstringspaces=false,          % underline spaces within strings only
  showtabs=false,                  % show tabs within strings adding particular underscores
  stepnumber=1,                    % the step between two line-numbers. If it's 1, each line will be numbered
  stringstyle=\color{mymauve},     % string literal style
  tabsize=4,                       % sets default tabsize to 2 spaces
}

\usepackage{etoolbox}
\renewcommand{\lstlistingname}{Diagram}% Listing -> Algorithm
\patchcmd{\thebibliography}{\chapter*}{\section*}{}{}

\author{John Karasinski}
\title{Homework 5}

\begin{document}
%\maketitle

\problem{}
\textit{Orbital Transfer Review: remind yourself: (discuss/justify decisions)}
\begin{enumerate}
\itemsep0em
\item Determine the $\Delta V$ required to move from a 200km coplanar parking orbit to the HST orbit
\item Determine the $\Delta V$ required to move from the ISS orbit to the HST orbit
\item Determine the $\Delta V$ required to deorbit from the HST orbit (must choose your de-orbit orbital params)
\end{enumerate}


\problem{}
\textit{Eclipse durations (text 5.3.2): an important design aspect of your solar array system is the relative durations of eclipse and insolation. Using the algorithm given in section 5.3.2,}
\begin{enumerate}
\itemsep0em
\item Compute the eclipse period for ISS, for HST, and for a typical GPS satellite (choose one)
\end{enumerate}

\problem{}
\textit{Let’s say we lose control of your spacecraft after it has undocked from HST, but before it has de-orbited.}
\begin{enumerate}
\itemsep0em
\item Estimate the orbital lifetime of your spacecraft (text 5.3.4) following loss of communications: assume HST
circular orbit, average solar activity.
\item Would it make any difference to the decay timescale whether your spacecraft was tumbling or not?
\end{enumerate}

\problem{}
\textit{Geostationary orbits (text 5.6):}
\begin{enumerate}
\itemsep0em
\item Using the linearized solution to Kepler’s equation given in eqns 5.27-5.29, plot ground-track fluctuations as
longitude vs latitude. Describe the results.
\item Define deadband and control limit-cycle in the context of GEO station keeping.
\item Consider a GEO satellite with nominal longitude of -100deg, and an onboard propellant system capable of
providing a total $\Delta V$ of 200m/s. For a maximum longitudinal error magnitude of 0.22deg, for how long can the satellite station-keep?
\end{enumerate}

\problem{}
\textit{Two spacecraft in elliptical Earth orbit with the orbital parameters as follows. Compute the relative position and velocity vectors.}
\begin{enumerate}
\itemsep0em
\item h= 52,059 km$^2$/s, e=0.0257240, i=60deg, $\Omega$=40deg, $\omega$=30deg, $\theta$=40deg
\item h= 52,362 km$^2$/s, e=0.0072696, i=50deg, $\Omega$=40deg, $\omega$=120deg, $\theta$=40deg
\end{enumerate}

\problem{}
\textit{Fly-around relative trajectories: for the lost EVA toolbox example considered in lecture, generate the relative motion}
plot for 1 orbital period, given initial conditions of:
\begin{enumerate}
\itemsep0em
\item Release relative velocity = (-0.1, 0, 0) m/s (prolate cycloid)
\item Release relative velocity = (0, 0, 0.1) m/s (ellipse)
\item Release relative velocity = (-0.1, 0, 0.1) m/s (initially 45deg backwards and up; describe subsequent motion)
\item For a and b, plot the trajectory with and without the nt$\ll$1 assumption. Discuss.
\item How about a release relative velocity = (0, 0.1, 0) m/s? Would you see the toolbox again or not?
\end{enumerate}

\problem{}
\textit{For your HST re-boost spacecraft, assume:}
\begin{itemize}
\itemsep0em
\item[--] Launch: drop-off circular orbit at 200km, in-plane with HST, 65deg phase angle behind HST - Phasing: 4-orbit phasing to point S1, 30km behind and 10km below HST
\item[--] Homing: Hohmann S1 to co-orbit waiting point S2, 1km behind HST
\item[--] Closing: Cycloid close waiting point S3, 200m behind HST
\end{itemize}
\begin{enumerate}
\itemsep0em
\item compute the required $\Delta V$ and elapsed time for each phase, and for the total rendezvous to S3
\item compute the view-angle to HST, measured from the orbit-tangent (for sensor acquisition)
\item plot the total quantitative relative motion (like Walter Fig. 8.26)
\end{enumerate}


\end{document}
