\documentclass[onecolumn,10pt]{jhwhw}

\usepackage{epsfig} %% for loading postscript figures
\usepackage{amsmath}
\usepackage{graphicx}
\usepackage{grffile}
\usepackage{pdfpages}
\usepackage{algpseudocode}
\usepackage{wrapfig}
\usepackage{pgfplots}
\usepackage{amsfonts}
\usepackage{booktabs}
\usepackage{siunitx}
\usepackage{commath}
\usepackage{rotating}
\usepackage{url}
\usepackage{multimedia}
\usepackage{hyperref}
\usepackage{mathtools}

% Default fixed font does not support bold face
\DeclareFixedFont{\ttb}{T1}{txtt}{bx}{n}{12} % for bold
\DeclareFixedFont{\ttm}{T1}{txtt}{m}{n}{12}  % for normal

% Custom colors
\usepackage{color}
\usepackage{listings}
\usepackage{framed}
\usepackage{caption}
\usepackage{bm}
\captionsetup[lstlisting]{font={small,tt}}

\definecolor{mygreen}{rgb}{0,0.6,0}
\definecolor{mygray}{rgb}{0.5,0.5,0.5}
\definecolor{mymauve}{rgb}{0.58,0,0.82}

\lstset{ %
  backgroundcolor=\color{white},   % choose the background color; you must add \usepackage{color} or \usepackage{xcolor}
  basicstyle=\ttfamily\footnotesize, % the size of the fonts that are used for the code
  breakatwhitespace=false,         % sets if automatic breaks should only happen at whitespace
  breaklines=true,                 % sets automatic line breaking
  captionpos=b,                    % sets the caption-position to bottom
  commentstyle=\color{mygreen},    % comment style
  deletekeywords={...},            % if you want to delete keywords from the given language
  escapeinside={\%*}{*)},          % if you want to add LaTeX within your code
  extendedchars=true,              % lets you use non-ASCII characters; for 8-bits encodings only, does not work with UTF-8
  frame=single,                    % adds a frame around the code
  keepspaces=true,                 % keeps spaces in text, useful for keeping indentation of code (possibly needs columns=flexible)
  columns=flexible,
  keywordstyle=\color{blue},       % keyword style
  language=Python,                 % the language of the code
  morekeywords={*,...},            % if you want to add more keywords to the set
  numbers=left,                    % where to put the line-numbers; possible values are (none, left, right)
  numbersep=5pt,                   % how far the line-numbers are from the code
  numberstyle=\tiny\color{mygray}, % the style that is used for the line-numbers
  rulecolor=\color{black},         % if not set, the frame-color may be changed on line-breaks within not-black text (e.g. comments (green here))
  showspaces=false,                % show spaces everywhere adding particular underscores; it overrides 'showstringspaces'
  showstringspaces=false,          % underline spaces within strings only
  showtabs=false,                  % show tabs within strings adding particular underscores
  stepnumber=1,                    % the step between two line-numbers. If it's 1, each line will be numbered
  stringstyle=\color{mymauve},     % string literal style
  tabsize=4,                       % sets default tabsize to 2 spaces
}

\usepackage{etoolbox}
\renewcommand{\lstlistingname}{Diagram}% Listing -> Algorithm
\patchcmd{\thebibliography}{\chapter*}{\section*}{}{}

\usepackage[utf8]{inputenc}
\usepackage{fourier}
\usepackage{array}
\usepackage{makecell}

% \renewcommand\theadalign{cb}
% \renewcommand\theadfont{\bfseries}
% \renewcommand\theadgape{\Gape[1pt]}
% \renewcommand\cellgape{\Gape[1pt]}

\author{John Karasinski}
\title{HST Rendezvous}

\begin{document}
%\maketitle

\chapter{Rendezvous Simulation}

Current orbital decay predictions place the Hubble Space Telescope (HST) at a nearly circular, 528 km altitude orbit during our proposed launch date of January 2020. After launch, we will enter a 200 km, in-plane parking orbit, then enter a 200 x 500 km phasing orbit. From our phasing orbit, we will perform a homing maneuver bring us to a holding point 15 km behind HST. When relative navigation sensors have acquired HST, we will perform a closing maneuver to bring us to 500 m below HST, after which an R-bar approach will be used to bring us to dock.

For the R-bar approach, our vehicle move up towards HST along its radial vector. Our vehicle will fire thrusters radially to close towards HST, and use small burns in the orbital velocity direction to negate the effects of orbital mechanics. If the R-bar is stopped at any point (in case of loss of communication or thruster failure, for instance), our vehicle will naturally move away from HST.

This approach has been adapted from the from the Space Shuttle's Optimized R-Bar Targeted Rendezvous (ORBT) profile~\footnote{Goodman, John L. ``History of Space Shuttle Rendezvous.'' (2011).}. ORBT was developed to optimally set up initial conditions for a low energy coast up the +R-bar. This profile was used from 1997 to the end of Space Shuttle program in 2011, lending more than a decade of operational flight heritage.

\section{How will we rendezvous, and how long does rendezvous take?}
\section{At what stage of the rendezvous are sensors active?}

\begin{table}[t!]
\begin{center}
\begin{tabular}{c c c c}
\toprule
Sensor & \# On-board & Range & Resolution \\
\midrule
Radar        & 2 & 100s of km - 100s of m & ? \\
LIDAR        & 2 & 10s of km - 2m & ? \\
Camera       & 2 & 100s of m - contact & ? \\
GPS          & 2 & - & 7.8 m~\footnote{gps.gov, http://www.gps.gov/systems/gps/performance/accuracy/} \\
\bottomrule
\end{tabular}
\end{center}
\caption{}
% \label{properties}
\end{table}

\begin{table}[t!]
\begin{center}
\begin{tabular}{c c c c}
\toprule
Sensor & \# On-board & Performance \\
\midrule
Startracker  & 2 & 0.005 deg \\
IMU          & 2 & 0.003 deg/hr \\
\bottomrule
\end{tabular}
\end{center}
\caption{}
% \label{properties}
\end{table}

\section{Can we recover from a single sensor failure at each stage?}

The most likely cause of failure for our spacecraft is a loss of attitude measurement. When the spacecraft is phasing with HST, its only sources of attitude measurement are startrackers and IMUs. Without frequent updates from the startrackers, however, the IMUs experience drift and quickly become useless. Once the spacecraft comes within a few 10s of km of HST, however, radar and LIDAR can calculate relative attitude. During final approach, the last few 100 m, stereo cameras can calculate sufficient pose estimation for docking. While there are multiple methods for relative attitude measurement, a lack of options for absolute attitude measurement is a weak point of our system.

Final docking requirements

\begin{table}[t!]
\begin{center}
\begin{tabular}{c c c c}
\toprule
Sensor & Displacement Error & Rate Error \\
\midrule
Lateral   & 11.4 cm & 1.3 cm/s \\
Range     & 20.3 cm & 5.1 cm/s \\
Roll      & 4 deg   & 1 deg/s \\
Pitch/Yaw & 4 deg   & 0.25 deg/s \\
\bottomrule
\end{tabular}
\end{center}
\caption{}
% \label{properties}
\end{table}


\section{How long can we stay at each part of the orbit?}
\section{Delta V worst case scenario?}

\end{document}


% \begin{table}[t!]
% \begin{center}
% \begin{tabular}{|c |c |c |c |c|}
% % \toprule
% \hline
% Maneuver & Time, MET & System & $\Delta$V, ft/sec & Duration, sec \\
% % \midrule
% \hline
% NC-1  & 00:05:27:28.9 & OMS (Both) & 98.0 & 59.2 \\ \hline
% NSR   & 01:03:43:59.7 & OMS (Both) & 49.8  & 30 \\ \hline
% NC-2  & 01:04:17:14.3 & OMS (Left) & 14.1 & 17.1 \\ \hline
% NC-3  & 01:17:55:30.1 & OMS (Right) & 12.1 & 14.8 \\ \hline
% % \bottomrule
% \end{tabular}
% \end{center}
% \caption{STS-61. The HST was grappled at 01:23:19:56 and berthed at 01:23:57:30.}
% % \label{properties}
% \end{table}


% \begin{table}[t!]
% \begin{center}
% \begin{tabular}{|c |c |c |c |c|}
% % \toprule
% \hline
% Maneuver & Time, MET & System & $\Delta$V, ft/sec & Duration, sec \\
% % \midrule
% \hline
% Reboost 1  & 06:16:59 & RCS Vernier (?) & - & 61 \\ \hline
% % \bottomrule
% \end{tabular}
% \end{center}
% \caption{STS-61. The HST was grappled at 01:23:28 and berthed at 2:00:10.}
% % \label{properties}
% \end{table}

% \begin{table}[t!]
% \begin{center}
% \begin{tabular}{|c |c |c |c |c|}
% % \toprule
% \hline
% Maneuver & Time, MET & System & $\Delta$V, ft/sec & Duration, sec \\
% % \midrule
% \hline
% NSR   & 01:03:36:21.9  & OMS         & 96  & 56.2 \\ \hline
% NC-2  & 01:05:02:00    & RCS Primary & 3.1 & 13.0 \\ \hline
% NH    & 01:17:06:04.9  & OMS         & 12  & 8.0  \\ \hline
% NC-3  & 01:17:53:00    & RCS Primary & 3.4 & 14.0 \\ \hline
% NPC-2 & 01:19:00:24    & RCS Primary & 0.5 & 2.0  \\ \hline
% NCC   & 01:20:07:46    & RCS Primary & 1.1 & 1.1  \\ \hline
% TI    & 01:21:07:52    & RCS Primary & 2.8 & 12.0 \\ \hline
% MC-1  & 01:21:34:16    & RCS Primary & 0.4 & 2.0  \\ \hline
% MC-2  & 01:22:02:40    & RCS Primary & 1.5 & 6.0  \\ \hline
% MC-3  & 01:22:12:40    & RCS Primary & 0.9 & 3.0  \\ \hline
% MC-4  & 01:22:22:40    & RCS Vernier & 0.2 & 1.0  \\ \hline
% % \bottomrule
% \end{tabular}
% \end{center}
% \caption{STS-82. The HST was grappled at 01:23:28 and berthed at 2:00:10.}
% % \label{properties}
% \end{table}

% \begin{table}[t!]
% \begin{center}
% \begin{tabular}{|c |c |c |c |c|}
% % \toprule
% \hline
% Maneuver & Time, MET & System & $\Delta$V, ft/sec & Duration, min:sec \\
% % \midrule
% \hline
% Reboost 1 & 04:01:09:28 & RCS Vernier & 6.6  & 20:41.9 \\ \hline
% Reboost 1A$_a$ & 04:06:07:04 & RCS Vernier & 3.3  & 10:12.6 \\ \hline
% Reboost 2 & 5:01:15:03 & RCS Vernier & 6.5  & 19:46.9 \\ \hline
% Reboost 3 & 07:01:33:00 & RCS Vernier & 10.4 & 31:53.5 \\ \hline
% % \bottomrule
% \end{tabular}
% \end{center}
% \caption{STS-82. a - Manuever required for space debris avoidance. The four reboost maneuvers raised the HST orbit an average of 8 nmi.}
% % \label{properties}
% \end{table}

% \begin{table}[t!]
% \begin{center}
% \begin{tabular}{|c |c |c |c |c|}
% % \toprule
% \hline
% Maneuver & Time, MET & System & $\Delta$V, ft/sec & Duration, sec \\
% % \midrule
% \hline
% NC-1 Trim & 01:03:42:23    & RCS Primary & 0.06 & 0.44   \\ \hline
% NC-2      & 01:04:36:49    & RCS Primary & 7.4  & 31     \\ \hline
% NSR Trim  & 01:17:36:55.1  & RCS Primary & 0.2  & 0.24   \\ \hline
% NCC       & 01:20:38:00    & RCS Primary & 0.3  & 1.0    \\ \hline
% TI        & 01:21:36:06    & RCS Primary & 4.1  & 8.7    \\ \hline
% MC-1      & 01:21:58:06    & RCS Primary & 0.3  & 0.5    \\ \hline
% MC-2      & 01:22:32:58    & RCS Primary & 0.9  & 2.9    \\ \hline
% MC-3      & 01:22:49:58    & RCS Primary & 0.4  & 1.0    \\ \hline
% MC-4      & 01:22:59:58    & RCS Vernier & 1.8  & 7.2    \\ \hline
% % \bottomrule
% \end{tabular}
% \end{center}
% \caption{STS-103. The HST was grappled at 01:23:44:01 and berthed at 2:00:52.}
% % \label{properties}
% \end{table}


% \begin{table}[t!]
% \begin{center}
% \begin{tabular}{|c |c |c |c |c|}
% % \toprule
% \hline
% Maneuver & Time, MET & System & $\Delta$V, ft/sec & Duration, sec \\
% % \midrule
% \hline
% NC2  & 000:17:50:50.5 & -X RCS & 4.5 & 19.7 \\ \hline
% NC3  & 01:02:55:32    & Multi-axis RCS & 3.1 & 12.6 \\ \hline
% NC4  & 01:17:47:01    & Multi-axis RCS & 4.8 & 20.4 \\ \hline
% NCC  & 01:18:38:57    & Multi-axis RCS & 1.3 & 5.5 \\ \hline
% MC-1 & 01:19:59:04    & Multi-axis RCS & 0.8 & 3.2 \\ \hline
% MC-2 & 01:20:34:27    & Multi-axis RCS & 0.4 & 1.79 \\ \hline
% MC-3 & 01:20:34:27    & +X RCS & 1.9 & 8.1 \\ \hline
% MC-4 & 01:21:01:28    & Multi-axis RCS & 1.9 & 8.1 \\ \hline
% % \bottomrule
% \end{tabular}
% \end{center}
% \caption{STS-109. The HST was captured at 001:21:09:19.}
% % \label{properties}
% \end{table}

% \begin{table}[t!]
% \begin{center}
% \begin{tabular}{|c |c |c |c |c|}
% % \toprule
% \hline
% Maneuver & Time, MET & System & $\Delta$V, ft/sec & Duration, min:sec \\
% % \midrule
% \hline
% Reboost 1 & 07:05:56:01.8 & RCS Vernier (?) & 11.8  & 36:00 \\ \hline
% % \bottomrule
% \end{tabular}
% \end{center}
% \caption{STS-109. The reboost maneuver raised the HST orbit an average of 3.6 nmi.}
% % \label{properties}
% \end{table}

% \begin{table}[t!]
% \begin{center}
% \begin{tabular}{|c |c |c |c |c|}
% % \toprule
% \hline
% Maneuver & Time, GMT & System & $\Delta$V, ft/sec & Duration, sec \\
% % \midrule
% \hline
% NC 1 & 131/21:49:54.65 & RCS & 19.5 & 90.88   \\ \hline
% NCC  & 133/13:41:50    & RCS &  1.6 & 7.0     \\ \hline
% MC 3 & 133/15:53:26.3  & RCS &  0.9 & 0.2     \\ \hline
% MC 4 & 133/16:03:26.5  & RCS &  2.1 & 8.9     \\ \hline
% % \bottomrule
% \end{tabular}
% \end{center}
% \caption{STS-125. The HST was captured at 133/17:30:52.}
% % \label{properties}
% \end{table}
